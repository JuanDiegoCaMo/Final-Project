\documentclass{article}
\usepackage[utf8]{inputenc}
\usepackage[spanish]{babel}
\usepackage{listings}
\usepackage{graphicx}
\graphicspath{ {images/} }
\usepackage{cite}

\begin{document}

\begin{titlepage}
    \begin{center}
        \vspace*{1cm}
            
        \Huge
        \textbf{Proyecto Final}
            
        \vspace{0.5cm}
        \LARGE
        Los primeros pasos
            
        \vspace{1.5cm}
            
        \textbf{Integrantes\\}
        \textbf{Juan Diego Cabrera Moncada}
            
        \vfill
            
        \vspace{0.8cm}
            
        \Large
        Departamento de Ingeniería Electrónica y Telecomunicaciones\\
        Universidad de Antioquia\\
        Medellín\\
        Marzo de 2021
            
    \end{center}
\end{titlepage}

\tableofcontents
\newpage
\section{Sección introductoria}\label{intro}
El videojuego que deseo desarrollar es del género de puzzle, de modo que la idea general del juego es plantear 3 niveles, los cuales tienen diferente dificultad, o mejor dicho lógica de solución, y cada uno cuenta con su propia escena. El juego está planteado para poner a prueba el analisis del jugador más que su habilidad para derrotar enemigos. De modo que el principal objetivo es ofrecer al jugador un problema compleja de resolver (No pueda ser resuelta de forma inmediata) \cite{Puzzle}.
\section{Desarrollo del juego}
Para ganar el nivel, el jugador debe dirigir su personaje hacia el objeto que representa la salida, es decir, una vez el personaje entra en contacto con dicho objeto, asumiendo que la salida ya haya sido desbloqueada, el jugador puede avanzar al siguiente nivel, dado caso que se trate del primer o segundo nivel, o terminar el juego si se trata del último nivel. El jugador cuenta con 3 vidas, si el jugador entra en contacto con un objeto que pueda restar una de ellas, se reinicia automáticamente el nivel y se descuenta una de sus vidas, una vez este contador llegue a cero, se redirecciona al jugador al primer nivel. El punto de partida en el cual inicia el personaje principal es el mismo para todas las escenas; no obstante, cada escena cuenta con sus propios puntos para ubicar el resto de objetos que la conforman.
\section{Escenas}
Previamente a proceder con la descripción de cada escena, hago énfasis en que cada escena corresponde a uno de los niveles del juego.
\subsection{Primera escena}

\subsection{Segunda escena}

\subsection{Tercera escena}

\section{Personajes}
\subsection{Personaje principal}
El personaje principal es un cuadrado que, según el poder que tenga, cambia su color y el símbolo que tiene en su interior. Los poderes no se tratan de objetos que hagan que el nivel sea más fácil sino que cambian la forma en la que el personaje se mueve. Se tiene pensado que sean las siguientes formas:
    - Movimiento vertical: El jugador solo puede mover a su personaje hacia arriba y hacia abajo. El cuadrado toma el siguiente aspecto:
    - Movimiento horizontal: El jugador solo puede mover a su personaje hacia la derecha y hacia la izquierda. El cuadrado toma el siguiente aspecto:
    - Movimiento dual: Esto se sabe desde el inicio del nivel, es decir, no puede ser adquirido ni hace referencia a la obtención de un poder a lo largo del nivel, se plantea el movimiento de dos cuadrados de modo que uno de ellos sólo es capaz de moverse en uno de los ejes (X ó Y) mientras el otro sólo puede moverse en el eje en el que el primer cuadrado no se puede mover.
    - Movimiento diagonal: El jugador solo puede usar una vez este poder, de modo que al adquirirlo, como a modo de disparo, debe elegir la diagonal a la que desea dirigirse, una vez elegida, el cuadrado cambia su posición dirigiéndose en la dirección elegida hasta encontrar un obstáculo con el cual colisione. Posterior a ello, retorna al tipo de movimiento que tenía antes de adquirir el poder de movimiento diagonal. El cuadrado toma el siguiente aspecto al tener este poder:
\subsection{Enemigos}
Los enemigos son todos aquellos objetos móviles o inmóviles en la capacidad de eliminar una de las vidas del personaje principal. Así, la interacción entre cualquier enemigo y el personaje principal se resume en que el personaje principal realiza su animación de muerte y pierde una de sus vidas, posterior a ello, dado caso que el jugador, antes de morir, contase con más de 1 vida, se reinicia la escena automáticamente a su estado inicial; de lo contrario, se redirige al jugador al primer nivel en su estado inicial.
Se tiene planteado así los siguientes enemigos:
    - Sierra circular: Este enemigo se traslada sobre un solo eje (Ya sea X ó Y), el cual está predefinido y no es posible cambiarlo por ningún método, de manera que al entrar en contacto con un muro cambia su dirección en el sentido contrario. Su interacción con el personaje principal para que pierda el jugador es por contacto directo con el mismo. Su imagen es la siguiente:
    - Torre láser: Este enemigo dispara un rayo láser cuyo ancho es igual al del cuadrado, de modo que dicho láser cubre todo el trayecto que recorre hasta colisionar con un muro de cualquier tipo. Su interacción con el personaje principal para que pierda el jugador es por contacto entre el láser y el personaje principal. Este enemigo puede ser móvil en un eje horizontal o vertical o inmóvil.
    - Láser: Es un objeto inmóvil que puede activar la consecuencia de la interacción entre un enemigo y el personaje principal al entrar en contacto con éste.
    - Púas: Estas púas se encuentran ocultas en un muro, de modo que sólo pueden dañar al personaje principal cuando éste se encuentre próximo al muro en el cual se ocultan las púas. Dichas púas tienen dos formas para salir del muro y dañar al personaje, y sólo pueden adquirir una de ellas: Pueden salir inmediatamente o segundos después de que el jugador active un botón en específico; o siguiendo un intervalo de tiempo previamente definido. Las púas pueden cubrir una serie de muros, no sólo uno en específico.
\section{Objetos}
\subsection{Power-Up Mov Horizontal}
Cuando el personaje principal entra en contacto con este objeto, el jugador obtiene el poder de moverse únicamente de manera horizontal de manera permanente. Su representación es la siguiente:
\subsection{Power-Up Mov Vertical}
Cuando el personaje principal entra en contacto con este objeto, el jugador obtiene el poder de moverse únicamente de manera vertical de manera permanente. Su representación es la siguiente:
\subsection{Power-Up Mov Diagonal}
Cuando el personaje principal entra en contacto con este objeto, el personaje principal se inmoviliza hasta que se seleccione la diagonal en la que se desea ir y, una vez seleccionada, se dirige a ésta hasta encontrar un muro. Es de único uso, por lo cual el personaje principal regresa a su método de movimiento anterior a obtener este poder una vez usado. Su representación es la siguiente:
\subsection{Muro}
Impide el movimiento a través de él de cualquier objeto móvil.
\subsection{Muro temporal}
Similar a un muro, pero éste sólo bloquea cualquier objeto móvil por intervalos de tiempo predeterminados o pueden ser desactivados por medio de un botón.
\subsection{Portal}
Consta de dos portales, para ser más específicos. Así, el jugador, al entrar en contacto con dicho portal cambia su posición desde el portal con el que entró en contacto hacia la posición del portal que completa el par. Ambos portales se pueden volver a usar, de modo que ambos pueden ser portales de ida y de llegada.
\subsection{Llave}
Este objeto es fundamental para desbloquear la puerta que representa la salida y término del nivel, si el jugador no la adquiere, al interactuar con el objeto "Puerta de Salida", ésta únicamente posee el mismo comportamiento que un muro. 
\subsection{Puerta de Salida}
Cuando el personaje principal interactúa con este objeto, dado caso que se cumpla la condición dada en el literal anterior, avanza al siguiente nivel, de lo contrario sólo actúa como un muro.
\bibliographystyle{IEEEtran}
\bibliography{references}

\end{document}
